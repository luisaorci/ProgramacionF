% Ejemplo de documento LaTeX
% Tipo de documento y tamaño de letra
\documentclass[10pt]{article}
% Preparando para documento en Español.
% Para documento en Inglés no hay que hacer esto.
\usepackage[spanish]{babel}
\selectlanguage{spanish}
\usepackage[utf8]{inputenc}
% EL titulo, autor y fecha del documento
\title{Compiladores e Interpretadores}
\author{Luisa Fernanda Orci Fernandez.}
\date{15 de Febrero de 2015}
% Aqui comienza el cuerpo del documento
\begin{document}
% Construye el título
\maketitle
\section{Introducción}
Un compilador es un programa informático que traduce un escrito de programación a un lenguaje de programación diferente, generando un programa que otra máquina pueda interpretar.\\
Un interpretador o interpréte, es un programa informático que puede analizar y ejecutar otros programas. La direfencia al compilador es que el interpretador solo realiza la traducción instrucción tras instrucción y no guarda el resultado de la traducción.\\

\section{Tabla comparativa entre algunos lenguajes de programación.}
\begin{tabular}{||l|l|l|l|l||}
\hline \hline
Nombre & Paradigma & Creadores & Año & Extención \\ \hline
C & Imperativo & Dennis M. Ritchie & 1972 & .c \\ \hline
C++ & Imperativo, orientado a objetos & Bjarne Stroustrup & 1983 & .cpp \\ \hline
Java & Orientado a objetos, imperativo & Sun Microsystems & 1995 & .java \\ \hline
Fortran & Imperativo & IBM & 1957 & .f90 (Depende/versión)\\ \hline
Python & Funcional, reflexivo, O.O & Guido van Rossum & 1991 & .py \\ \hline
Ruby & O.O, reflexivo & Yukihiro Matsumoto & 1995 & .rb \\ \hline \hline
\end{tabular}
\section{A continuación se muestran ejemplos de estos lenguajes de programación; mediante el juego ``Adivina el número''}
\subsection{C}
\begin{verbatim}
/* Juego en C, Ansi-Style */
#include <stdio.h>
#include <stdlib.h>
#include <unistd.h>

int main(void)
{
puts("Hola! Trataré de adivinar un número.");
puts("Piensa un número entre 1 y 10.");
sleep(5);
puts("Ahora multiplícalo por 9.");
sleep(5);
puts("Si el número tiene 2 dígitos, súmalos entre si: Ej. 36 -> 3+6=9. Si tu número tiene un solo dígito, súmale 0.");
sleep(5);
puts("Al número resultante súmale 4.");
sleep(8);
puts("Muy bien. El resultado es 13 :)");
return(EXIT_SUCCESS);
}
\end{verbatim}

\subsection{C++}
\begin{verbatim}
#include <iostream>
#include <unistd.h>

int main()

{
  std::cout << "Hola! Trataré de adivinar un número. Piensa en un número entre 1 y 10\n";
  sleep(5);
  std::cout << "Ahora multiplícalo por 9.\n";
  sleep(5);
  std::cout << "Si el número tiene 2 dígitos, súmalos entre si: Ej. 36 -> 3+6=9. Si tu número tiene un solo dígito, súmale 0.\n";
  sleep(5);
  std::cout << "Al número resultante súmale 4.\n";
  sleep(8);
  std::cout <<"Muy bien. El resultado es 13 :)\n";
  return(0);
}

\end{verbatim}

\subsection{Fortran90}
\begin{verbatim}
program juego
  write(*,*) 'Hola! Trataré de adivinar un número.'
  write(*,*) 'Piensa un número entre 1 y 10.'
  call sleep(5)
  write(*,*) 'Ahora multiplícalo por 9.'
  call sleep(5)
  write(*,*) 'Si el múmero tiene 2 dígitos, súmalos entre si: Ej. 36 -> 3=6+9. Si tu número tiene un solo dígito, súmale 0.'
  call sleep(5)
  write (*,*) 'Al número resultante súmale 4.'
  call sleep(8)
  write (*,*) 'Muy bien. El resultado es 13 :)'
end program juego

\end{verbatim}

\subsection{Java}
\begin{verbatim}
// Juego en Java
class juego {
    static public void main( String args [] ) {
	System.out.println("Hola! Trataré de adivinar un número.");
System.out.println("Piensa un número entre 1 y 10.");
try { 
Thread.sleep(5000);
} catch(InterruptedException ex) {
Thread.currentThread().interrupt();
}
System.out.println("Ahora multiplícalo por 9.");
try {
Thread.sleep(5000);
} catch(InterruptedException ex) {
Thread.currentThread().interrupt();
}
System.out.println("Si el número tiene 2 dígitos, súmalos entre si: Ej. 36 -> 3+6=9. Si tu número tiene un solo dígito, súmale 0.");
try {
Thread.sleep(5000);
} catch(InterruptedException ex) {
Thread.currentThread().interrupt();
}
System.out.println("Al número resultante súmale 4.");
try {
Thread.sleep(8000);
} catch(InterruptedException ex) {
Thread.currentThread().interrupt();
}
System.out.println("Muy bien. El resultado es 13 :)");
}

}
\end{verbatim}

\subsection{Python}
\begin{verbatim}
# Juego en Python
# -*- coding: 850 -*-
import time
print "Hola! Trataré de adivinar un número."
print "Piensa un número entre 1 y 10."
import time
time.sleep(5)
print "Ahora multiplícalo por 9."
import time
time.sleep(5)
print "Si el número tiene 2 dígitos, súmalos entre si: Ej. 36 -> 3+6=9. Si tu número tiene un solo dígito, súmale 0."
import time
time.sleep(5)
print "Al número resultante súmale 4."
import time
time.sleep(8)
print "Muy bien. El resultado es 13 :)."
\end{verbatim}

\subsection{Ruby}
\begin{verbatim}
# -*- coding: utf-8 -*-
# Juego en Ruby
#encoding: utf-8
puts "Hola! Trataré de adivinar un número."
puts "Piensa un número entre 1 y 10."
sleep(5)
puts "Ahora multiplícalo por 9."
sleep(5)
puts "Si el número tiene 2 dígitos, súmalos entre si: Ej. 36 -> 3+6=9. Si tu número tiene un solo dígito, súmale 0."
sleep(5)
puts "Al número resultante súmale 4."
sleep(8)
puts "Muy bien. El resultado es 13 :)."
\end{verbatim}

\end{document}
