% Ejemplo de documento LaTeX
% Tipo de documento y tamaño de letra
\documentclass[12pt]{article}

% Preparando para documento en Español.
% Para documento en Inglés no hay que hacer esto.
\usepackage[spanish]{babel}
\usepackage[utf8]{inputenc}
\usepackage{longtable}

% EL titulo, autor y fecha del documento
\title{Manual breve sobre los comandos de Bash}
\author{Luisa Fernanda Orci Fernandez}
\date{04 de Febrero del 2015}

% Aqui comienza el cuerpo del documento
\begin{document}
\maketitle
% Construye el título
\begin{longtable}{|p{1 in}|p{3.0 in}|p{2 in}|}
\hline
Comando & Descripci\'on & Ejemplo \\
\hline
echo SHELL & Te muestra el shell que est\'as utilizando. & /bin/bash \\
flecha hacia arriba y hacia abajo & Se utilian para navegar en el historial de los comandos utilizados, es un atajo. & arriba, abajo \\

pwd & Son las siglas de Present Working Directory, se refiere al directorio en el que est\'as actualmente & pwd \\

ls & Da una lista de todo el contenido de un directorio & ls \\

ls -a & Da una lista de todos los archivos de un directorio incluyendo los archivos escondidos & ls -a \\

cd & Abrebiaci\'on para Cambio de Directorio, te mueve del directorio donde est\'as a otro & cd Notas \\

file & Te da la informaci\'on sobre el archivo o directorio en el cual est\'as & file notas.txt \\

man comando & Busca el comando en el manual & man bash \\

man -k & Palabra que se est\'a buscando, busca la palabra o termino en el manual & man -k shell \\

mkdir & Se utiliza para crear un nuevo directorio, (make directory en ingles) & mkdir Fotos \\

rmdir & Se utiliza para eliminar un directorio, (remove directory en ingles) & rmdir Fotos \\

touch & Crea un archivo en blanco & touch Libro \\

cp & Se utiliza para copiar una foto o un directorio, se pone primero el comando, seguido del archivo o directorio que se desea copiar, seguido del nombre del directorio o archivo donde se va a pegar & cp Fotos fotos \\

mv & Se utiliza para renombrar archivos y directorios & mv fotos pictures \\

rm & Borra un archivo, (en ingels remove) & rm Libro \\

vi & Se utiliza para editar un archivo & vi notas.txt \\

cat & Se utiliza para ver un archivo & cat notas.txt \\

less & Se utiliza para ver archivos grandes de una forma mas ordenada & less notas.txt \\

chmod & Cambia los permisos de un archivo o de un directorio & chmod Notas 777 \\

ls -ld & Te muestra los permisos de un directorio en particular & ls -ld Notas \\

grep & Nos permite buscar dentro de nuestros archivos & grep cadena archivo \\

$|$ (pipe) & Sirve para que la salida de un archivo se redirija a la entrada de otro & grep nose notas.txt $|$ gzip $>$ archivo \\

kill & Mata un proceso & kill 309 \\

echo & Imprime mensajes en la pantalla & echo hola \\

date & Te da el dia, la fecha y la hora & Date
Tue Jan 27 19:30:30 MTS 2015 \\ \hline
\end{longtable}



% Nunca debe faltar esta última linea.

\end{document}
